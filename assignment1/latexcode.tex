\documentclass{article}
\usepackage[utf8]{inputenc}
\usepackage{mathtools}
\title{PROBABILITY PROBLEM 1}
\author{ai21btech11028 }
\date{March 2022}

\begin{document}

\maketitle
\begin{center}\textbf{Problem Statement}\end{center}\\
\begin{center}Question 1.C of ICSE maths 2014 paper\end{center}\\
\begin{center}A die has 6 faces marked by given numbers as shown below\end{center}\\
\begin{center}\framebox{1}\hspace{0.3cm}\framebox{2}\hspace{0.3cm}\framebox{3}\hspace{0.3cm}\framebox{-1}\hspace{0.3cm}\framebox{-2}\hspace{0.3cm}\framebox{-3}\end{center}\\
\begin{center}The die is thrown once. What is the probability of getting:\end{center}\\
(i)   a positive integer\\
(ii)  an integer greater than -3\\
(iii)  the smallest integer\\


\section{Given:}
A die has six faces marked by the numbers 1,2,3,-1,-2,-3.\\
The die is thrown once.
\section{To Find:}
\begin{itemize}
    \item The probability of getting a positive integer.
    \item The probability of getting an integer greater than -3.
    \item The probability of getting the smallest integer.
\end{itemize}
\section{Solution: }
\subsection{part 1:}
Let S be the sample space.\\
S=\{1,2,3,-1,-2,-3\}\\
Thus  ,   n(S) = 6\\
Let E1 be the event of getting positive integer.\\
E1 = \{1,2,3\}\\
Thus  ,  n(E1) = 3\\
Probability   P(E1) = n(E1)/n(S)\\
Thus   p(E1) =3/6\\
             =1/2\\
             =0.5
\subsection{part 2:}
Let S be the sample space.\\
S =\{1,2,3,-1,-2,-3\}\\
Thus  ,   n(S) = 6\\
Let E2 be the event of getting an integer greater than -3.\\
E2 = \{1,2,3,-1,-2\}\\
Thus  ,  n(E2) = 5\\
Probability   P(E2) = n(E)/n(S)\\
Thus   p(E2) =5/6\\
             =0.833\\
\subsection{part 3:}
Let S be the sample space.\\
S =\{1,2,3,-1,-2,-3\}\\
Thus  ,   n(S) = 6\\
Let E3 be the event of getting the smallest integer.\\
E3 = \{-3\}\\
Thus   ,  n(E3) = 1\\
Probability   P(E3) = n(E3)/n(S)\\
Thus   ,  p(E3) = 1/6\\
=0.166\\